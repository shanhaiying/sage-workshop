\documentclass{amsart}
\usepackage{sagetex}


\begin{document}

\title{Sage\TeX\ on cloud.sagemath.com is interesting!}
\author{TJ Hitchman}
\date{\today}

\maketitle



Big testing. What happens when I update this?
\begin{center}
\sageplot[width=.75\textwidth]{plot(sin(x), (x,-4,4), color='red')}
\end{center}


\begin{sageblock}
#This will display the code.
A = matrix(ZZ, 2,2, [[2,1],[1,1]])
a = A.det()
\end{sageblock}


Of course, passing this to \texttt{Sage} gives us the output: $\sage{a}$. This kind of thing can be helpful when we want to set up a problem without solving it first.

\begin{sagesilent}
num = 234
\end{sagesilent}

The factorization of $\sage{latex(num)}$ is $\sage{latex(num.factor())}$.

Also, we can do displayed math.
\[
\sum_{n=1}^{1000} \frac{1}{n^2} \approx \sage{sum( 1/(x^2), x, 1, 1000).n()}
\]


\end{document}
%sagemathcloud={"zoom_width":100}