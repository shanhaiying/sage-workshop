\documentclass{beamer}
\usepackage{url}
\begin{document}

\title[Ways to Sage]{Ways to Use Sage}
\author{Theron~J~Hitchman}
\institute{Department of Mathematics\\ University of Northern Iowa\\ Cedar Falls, Iowa}

\date[Augustana Workshop]{14 November 2013}

\frame{\titlepage}

% official statement

\begin{frame}
\frametitle{Ways to Use Sage}

The Official Sage Documentation List (for local installs):
{\tiny
\url{http://www.sagemath.org/doc/tutorial/introduction.html}}

\begin{itemize}
\item Notebook Graphical Interface
\item Interactive Command Line
\item Compiled and Interpreted Programs
\item Stand-alone scripts
\end{itemize}

And nowadays, you can use can use the cloud.sagemath.com service, or the single cell server!
\end{frame}

% transition question

\begin{frame}

\frametitle{The Big Question}

What about these choices in an educational context?
\end{frame}


% local install

\begin{frame}
\frametitle{Local Install}

\begin{block}{Benefits}
\begin{itemize}
\item Speed
\item Flexibility
\end{itemize}
\end{block}

\begin{block}{Challenges}
\begin{itemize}
\item Installation and upkeep is up to \emph{each user}
\item sharing files
\end{itemize}
\end{block}

\end{frame}

% campus notebook server

\begin{frame}
\frametitle{Semi-Local: A campus notebook server}

\begin{block}{Benefits}
\begin{itemize}
\item Easier for student to try
\item Sharing features are built in: sharing is at the worksheet level; publishing is available
\end{itemize}
\end{block}

\begin{block}{Challenges}
\begin{itemize}
\item Installation and upkeep is up to \emph{some user}
\item filesystem is rudimentary
\item students find a way to use the wrong server
\item system not designed for "large" numbers of users. resource intensive on hardware (public server is often overloaded)
\end{itemize}
\end{block}

\end{frame}

%  the sage cell server

\begin{frame}
\frametitle{The Sage Cell Server}

\begin{block}{Benefits}
\begin{itemize}
\item Extremely easy to use (no sign in!)
\item Sharing features are built in: share urls only
\item can be embedded in \emph{any} web page
\item serious work underway to integrate with WeBWorK
\end{itemize}
\end{block}

\begin{block}{Challenges}
\begin{itemize}
\item No installation or upkeep
\item single computations only
\item access to "help" and "documentation" is in progress.

NO [tab] completion
\end{itemize}
\end{block}

\end{frame}

%  global solution: Cloud service

\begin{frame}
\frametitle{The Global Solution: Cloud service}

\begin{block}{Benefits}
\begin{itemize}
\item installation and upkeep are managed
\item POWER
\item Sharing features are built in: sharing is at the project level; public projects are ``in progress"
\end{itemize}
\end{block}

\begin{block}{Challenges}
\begin{itemize}
\item overwhelming freedom for students without any skills
\item It is in beta stage, and still undergoing some development.
\end{itemize}
\end{block}

\end{frame}

\begin{frame}

\begin{block}{A message from William Stein}
If they ask about the possibility of commercial support/hosting --i.e., so that they don't have to worry about install/backups/etc.-- let them know that I'm looking for business model(s) for \url{https://cloud.sagemath.com}, and I'm betting I could help "a college that is considering adopting Sage full scale", and at a price that is much cheaper than Mathematica site licenses, since Sage + cloud = much more efficient than 600-person company living off boxed software.
\end{block}
\end{frame}



\end{document}

%sagemathcloud={"zoom_width":100}